\section{Methods}
\label{sec:methods}
% NOT finished

\subsection{Preprocess Pipeline}
\label{subsec:preproc}
% NOT finished

After a careful study of spectral characteristics of heart murmurs from medical literature \cite{Donnerstein_1989, Noponen_2007}, and with reference to previous work \cite{Schmidt_2010}, we constructed the PCG signal preprocess pipeline as follows:
\begin{itemize}
    \item Resample to 1000 Hz;
    \item Butterworth bandpass filtering of order 3 and cutoff frequencies 25 - 400 Hz;
    \item Z-score normalization to zero mean and unit variance.
\end{itemize}

\subsection{Neural Network Backbones}
\label{subsec:nn}
% NOT finished

Inspired by the work of \texttt{wav2vec2} \cite{baevski2020wav2vec}, and under the consideration of exploring and utilizing the powerfulness of (self-supervised) pretrained models \cite{wolf-etal-2020-transformers}, we adopted a shrunken \texttt{wav2vec2} as one of our neural network backbones. ...

Since PCG signals share similar physiological origin as electrocardiogram (ECG) signals, we further adjusted and tested several neural network backbones \cite{Kang_2022_cinc2021_iop} that have proven effective in ECG research problems. ...

\subsection{Multi-Objective Learning}
\label{subsec:mol}
% NOT finished

The challenge has 2 tasks, namely ... The challenge dataset \cite{Oliveira_2021_CirCor} further provides heart sound segmentation annotations.

\cite{Caruana_1997_mtl}

\subsection{Training Setups}
\label{subsec:training}
% NOT finished

to write


\subsection{Demographic Features}
\label{subsec:demo_feat}
% NOT finished

to write
