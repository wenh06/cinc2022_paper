\section{Methods}
\label{sec:methods}
% NOT finished

\subsection{Preprocess Pipeline}
\label{subsec:preproc}
% NOT finished

After a careful study of spectral characteristics of heart murmurs from medical literature \cite{Donnerstein_1989, Noponen_2007}, and with reference to previous work \cite{Schmidt_2010}, we constructed the PCG signal preprocess pipeline as follows:
\begin{itemize}
    \item Resample to 1000 Hz;
    \item Butterworth bandpass filtering of order 3 and cutoff frequencies 25 - 400 Hz;
    \item Z-score normalization to zero mean and unit variance.
\end{itemize}

\subsection{Neural Network Backbones}
\label{subsec:nn}
% NOT finished

Inspired by the work of \texttt{wav2vec2} \cite{baevski2020wav2vec}, and under the consideration of exploring and utilizing the powerfulness of pretrained models \cite{wolf-etal-2020-transformers}, we adopted a shrunken \texttt{wav2vec2} as one of our neural network backbones and used the time-domain signals, namely the PCG waveforms, as model input, rather than the derived time-frequency-domain signals, for example, the spectrogram. Since PCG signals have significantly lower sampling frequencies compared to conventional human voice audio signals, we reduced the dimension (number of channels) of the encoder of the `wav2vec2` model, as well as its depth (number of hidden layers).

Considering that PCG signals share a similar physiological origin as electrocardiogram (ECG) signals, we further adjusted and tested several neural network backbones \cite{Kang_2022_cinc2021_iop} that have proven effective in ECG research problems, including multi-branch convolutional neural networks (CNNs), SE-ResNets, TResNets, etc. We enlarged the kernel sizes of each convolution in these backbones by a factor of 2, which is the ratio of the sampling frequencies.

\subsection{Multi-Objective Learning}
\label{subsec:mol}
% NOT finished

The challenge has 2 tasks, namely
\begin{itemize}
    \item the prediction of the existence of heart murmurs into 3 classes: ``Present'', ``Unknown'', ``Absent'';
    \item the prediction of clinical outcome of the subject into 2 classes: ``Abnormal'', ``Normal''.
\end{itemize}

... The challenge dataset \cite{Oliveira_2021_CirCor} further provides heart sound segmentation annotations.

\cite{Caruana_1997_mtl}

\subsection{Training Setups}
\label{subsec:training}
% NOT finished

to write


\subsection{Demographic Features}
\label{subsec:demo_feat}
% NOT finished

to write
